\documentclass{article}
\usepackage[french]{babel}
\usepackage[utf8]{inputenc}

\usepackage{graphicx}
\graphicspath{{image/}}
\begin{document}
\title{Git}

\author{François}
\maketitle

Git is local, github est sur le net.\\
\verb!git add -u! va mettre à jour l'index des fichiers qui ont changé de nom où qui ont été supprimés \\
\verb!git commit -m «message»! pour mettre à jour le repo local \\
Pour créer une branche, taper \\
\verb!git checkout -b branchname!\\
Pour voir sur quelle branche :\\
\verb!git branch!\\
Pour changer de branche :\\
\verb!git checkout master! qui va repasser sur la branche \verb!master! dans ce cas.\\
\\
Pour connecter un repository à un repository en ligne :\\
\verb!git remote add origin https://github.com/daillouf/RSAmacro!\\

Pour copier «cloner» un repository :\\
\verb!git clone URL!

Après avoir cloner un repository qui est un fork, en ligne, on peut, en ligne de commande, le «connecter» au dossier forké original : \\
\verb!git remote add upstream URL_REPO_ORIGINAL!\\


\end{document}